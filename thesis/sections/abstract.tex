\newenvironment{poliabstract}[1]
   {\renewcommand{\abstractname}{#1}\begin{abstract}}
   {\end{abstract}}
 \providecommand{\keywords}[1]
{
    \small	
    \textbf{\textit{Keywords -- }} #1
}


\selectlanguage{english}
\begin{poliabstract}{Abstract} 
    Semantic image segmentation is one of the tasks of computer vision that is crucial for understanding and interpreting images. There are a lot of different methods aimed to perform it, with the majority being based on machine learning. The primary purpose of this study was to examine the possibility of developing an image semantic segmentation system with use of Conditional Random Fields. Multiple different algorithms were studied and tested, in order to provide a method capable of fulfilling the thesis purpose. In this study, three main procedures were developed. Firstly, images were modelled with the use of factor graphs, which represented image features and labelling outcomes, as well as the relations between them. Feature selection was obtained by a stepwise regression process, while a feature function was based on an estimation of the probability density function of all features and available labels. For parameter learning Stochastic Gradient Descent was chosen, which included few simplifications aimed to make the computations feasible and less resource consuming. The final prediction of labelling of unknown images was performed by an inference process with use of Loopy Belief Propagation. The developed method correctly segmented objects in an image that they were differing only by colours if CIELAB colour space was used. By incorporating the contextual data to the model it was also possible to distinguish objects by shape. A large improvement of segmentation, especially in the presence of noise, was obtained by including pairwise relations between regions in a prediction process.
    
    \keywords{semantic image segmentation, Conditional Random Fields, Structured Prediction, factor graphs, Stochastic Gradient Descent, Loopy Belief Propagation }
\end{poliabstract}

 
%\selectlanguage{polish}
%\begin{poliabstract}{Streszczenie}
%   \lipsum[1]
%\end{poliabstract}
%\selectlanguage{english}