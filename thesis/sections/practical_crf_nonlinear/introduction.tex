This chapter will be devoted to the description of the second part of the created system. It will put an emphasis on differences between this part, and the part described in \textit{chapter \ref{chapter:linear}: \nameref{chapter:linear}}. The main difference lies in a definition of the feature function, which in this chapter was inspired by an article "TextonBoost for Image Understanding:
Multi-Class Object Recognition and Segmentation by
Jointly Modeling Texture, Layout, and Context" \cite{article_main} written by Jamie Shotton et al. A more complex feature function is needed as the goal of this part of the system involves detection of objects based on their shape, and not only an their colour.

The first section of this chapter will be solely devoted to an explanation of how the feature function was defined. Furthermore, this chapter will contain a detailed description on how the probability distribution of labels and features available in the dataset was estimated. The next two sections will be devoted to the an overview of the experimental part. There were two different types of experiments conducted - semantic segmentation on noise free images, and on noised images. For each type a generated dataset generated will be presented. This part of the system involves an automatic process of feature selection. An explanation of how it works, and which features were chosen for a specific experiment will be also provided in this chapter.  