After the process of feature selection was completed parameter training by means of Stochastic Gradient Descent could take place. Values of the trained weights are presented in table \ref{table:weights_nonlinear_noise_free}.
\npdecimalsign{.}
\nprounddigits{5}
\begin{table}[ht]
\caption{Trained weights for segmentation of noise free images.}
\centering
\begin{tabular}{|c|c|c|}
\hline
\rowcolor[HTML]{C0C0C0} 
$w_1$(unary potential) & $w_{2,1}$ (pairwise potential) & $w_{2,2}$ (bias) \\ \hline
?? & ?? & ?? \\ \hline
\end{tabular}
\label{table:weights_nonlinear_noise_free}
\end{table}

Having a parametrised model it was possible to perform inference on the set of unknown test samples.
