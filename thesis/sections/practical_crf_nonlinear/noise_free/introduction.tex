The first experiment that was conducted for this part of the system concerned semantic segmentation of noise-free images. The main goal was to differentiate red objects in a green neighbourhood, which differ only in shape. Before the training and inference process could take place, hyperparameter values had to be defined. 
\begin{table}[ht]
    \caption{Values of hyperparameters required for the experiment on noise free images.}
    \centering
    \begin{tabular}{|l|c|}
        \hline
        \rowcolor[HTML]{C0C0C0} 
        \textbf{Parameter name} & \textbf{Parameter value} \\ \hline
        Number of images (inputs) & 800 \\ \hline
        Number of states (outputs) & 4 \\ \hline
        Number of superpixels & 500 \\ \hline
        Training step & ?? \\ \hline
        Regularisation factor & ?? \\ \hline
        Number of training epochs & 100 \\ \hline
        Convergence tolerance & 0.1 \\ \hline
        Number of histogram bins & 17 \\ \hline
        Grid & $7 \times 7$ \\ \hline
        Neighbourhood size & 3 \\ \hline
    \end{tabular}
    \label{table:hyperparameters_nonlinear_noise_free}
\end{table}
Table \ref{table:hyperparameters_nonlinear_noise_free} shows a list of hyperparameters that were required in the experiment on noise-free images together with their chosen values.