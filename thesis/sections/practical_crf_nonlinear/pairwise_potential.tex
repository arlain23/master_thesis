The next component that is needed to compute an energy of the whole system is a pairwise potential. While unary potential was responsible for detecting shapes, the role of the pairwise potential is to remove noises from images. For this part of the system, feature function based on Potts model was used. The general definition of this method was already presented in \textit{section \ref{sec:pariwise_potential}: \nameref{sec:pariwise_potential}}. Computation of the pairwise potential is based on differences between features of neighbouring superpixels from the original image, and not from the image after the process of colour quantisation as it was in case of the unary potential. A chosen feature function returns a vector of two elements, as in equation \ref{eq:nonlinear_potts_model}.
\begin{equation}
    \label{eq:nonlinear_potts_model}
    \varphi(y_i,y_j) = \begin{bmatrix}
        \exp{(-\beta \left \| \phi(x_i) - \phi(x_j)\right \|^2)} \\
        1
    \end{bmatrix}
\end{equation}
The second element of this vector is a unit term needed for bias, and the first one defines similarity between neighbouring superpixels in terms of a feature difference $ \left \| \phi(x_i) - \phi(x_j)\right \|$. This feature is modelled as a three-dimensional vector of colours in CIELAB colour space. The difference between two colours is computed in terms of Euclidean distance as in equation \ref{eq:colour_distance}. 
\begin{equation}
    \label{eq:colour_distance}
    \left \| \phi(x_i) - \phi(x_j)\right \|=\sqrt {(L_i-L_j)^2+(a_i-a_j)^2+(b_i-b_j)^2}
\end{equation}
Value of the parameter $\beta$ is calculated basing on the mean colour differences between pixels for a given image.

Hence, knowing the definitions of the feature function for unary and pairwise potentials it is possible to perform the process of the process of parameter learning. As the unary potential is defined by just one number, representing the probability of occurrence of a given label conditioned on superpixels features, and the pairwise potential is in a form of a two-dimensional vector, there are three weights to be learned by Stochastic Gradient Descent method. Those weights model the importance of both types of potentials for a given problem. Then, with a trained system it is possible to perform the experiments by means of inference.