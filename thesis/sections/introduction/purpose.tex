The topic of this thesis lies in the area of machine learning, which is a subset of artificial intelligence aimed to analyse known data, identify its patterns and trends, and by their generalisation make predictions about new data, without being programmed in an explicit way. The aim of machine learning is to mimic the learning ability of a human brain and apply it for various tasks which require a decision-making process. Applications of machine learning are numerous, and many of them are present in our daily life. An example would be Social Media Services which use machine learning for targeted advertisements, proposals of new friends or for recommendations of potentially interesting groups, sites and products. It is also used in traffic predictions, spam filters, online customer support bots or virtual personal assistants like Siri or Alexa, and in a number of different applications that have hundreds of thousands of daily users. Apart from purely commercial usages, machine learning has also a huge impact on currently performed research in a number of different scientific areas such in a medical field, to better adjust treatments or locate tissue pathologies, or physics, chemistry and astronomy for performing simulations that aid in finding distant planets or new particles \cite{ml_uses}. 

Regardless of an application, any machine learning task needs data to learn from. The training data may be expressed as simple numbers but very often it has a more structured form of for instance images. Semantic image segmentation is one of the main tasks in automatic image recognition systems. It is aimed to group together parts of an image that belongs to the same object class, which carry some semantic meaning. Scene understanding is a necessary part of many technologies that may now be still under development but presumably will become common in a near future, with the most prominent example of autonomous driving. Semantic image segmentation can be tackled with various different methods, among others by using Conditional Random Fields. Thus, the main purpose of this thesis is to develop a system that will be capable of performing semantic image segmentation using this method.
