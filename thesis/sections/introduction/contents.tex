This dissertation will be divided into two main parts, firstly all the theoretical concepts needed to understand the process of semantic image segmentation with use of Conditional Random Fields will be provided, and next there will be a description of the created system and the performed experiments.

The explanation of the main goal of thesis will be presented in 
\textit{Chapter \ref{chapter:segmentation}: \nameref{chapter:segmentation}}. It will contain a description of what the semantic image segmentation is, with which methods it can be achieved and what are its possible applications. Next, \textit{chapter \ref{chapter:structured_prediction}: \nameref{chapter:structured_prediction}} will provide the theory behind the core of the created system in terms of the procedures and algorithms required to fulfil its purpose. This chapter will begin with an explanation on how the input data are modelled for the further processing. Then, the second part of the topic of this thesis will be described, which are Conditional Random Fields. Next section will be devoted to providing information on how the final prediction on an unknown sample is obtained. After that, there will be an explanation on the machine learning part of the system, meaning how the parameters of the model are learned from the known dataset. Last section will present a way in which images are transformed into data understandable by that model. 

\textit{Chapter \ref{chapter:system}: \nameref{chapter:system}} will provide an introduction on what are the goals of the developed system. Moreover, the implementation details in terms of the technological stack will be included and the key components of the system will be briefly described. \textit{Chapters \ref{chapter:linear}: \nameref{chapter:linear}} and \textit{\ref{chapter:nonlinear}: \nameref{chapter:nonlinear}} will be devoted to the experimental part of this dissertation. They will include a description of what the system is supposed to do and what are the steps that needed to be implemented. Furthermore, those chapters will contain information on which algorithms were chosen for the specific tasks, how the image dataset is transformed into meaningful data and finally what are the results of the semantic segmentation process. Last chapter \textit{\nameref{chapter:conclusions}} will be devoted to the summary and comparison of the presented results, as well as, a description on possible improvements and extensions to the created system. 