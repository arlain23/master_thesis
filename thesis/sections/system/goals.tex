The primary goal of the developed system is to perform semantic segmentation on images, however, the way in which this final task is achieved is equally important to the results that are to be presented. The created system may be divided into two interconnected parts aimed to perform the segmentation in a slightly different manner. The components of both parts of the system are exactly the same and the only difference, and yet a substantial one, lies in a way a feature function needed for energy calculation is defined. The first part, which will be described in details in \textit{chapter \ref{chapter:linear}: \nameref{chapter:linear}} is based on the simplest way of expressing features, which is by using a feature function that directly outputs a feature vector composed of low-level features image pixels. On the other hand, the feature function used in the second part of the system involves a higher level of abstraction as instead of directly operating on features, a probability estimator is introduced that outputs a probability of a given label conditioned on a set of pixel features. The details of this part of the system will be provided in \textit{chapter \ref{chapter:nonlinear}: \nameref{chapter:nonlinear}}.