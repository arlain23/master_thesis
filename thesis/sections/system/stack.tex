From all the presented components only the preprocessing stage was implemented with the use of an external library, which was delivered by the supervisor of this thesis. Other components were written from scratch only with the use of some basic Java libraries that facilitate coding and testing such as Apache Commons, log4j or JUnit. As the preprocessing component was incorporated to the system after some progress in implementation of other components had already been made, during the initial planning phase there was no restrictions on the programming tools that are to be used. That is why the technological stack was mainly the matter of personal choice. A language chosen for implementation of this system was Java 1.8. It is a well established language that is platform independent. This is a huge advantage in situations in which the same program is to be developed on two different operating systems, or if the development and testing phases happens on different machines. Furthermore, according to TIOBE Programming Community index \cite{tiobe}, which provides data on language popularity, Java is currently the most popular language and it has been in top three for nearly 20 years. It also has a large availability of free tools, including Eclipse IDE, which was used for development of the system presented in this dissertation. 

When it comes to the hardware on which all the phases of the system development took place, it was a personal computer with processor Intel Core i5-5250U with 8GB of RAM and Windows 10 as an operating system.