The primary goal of the system described in this dissertation is to perform semantic segmentation on images. However, they way in which this final task is achieved is equally important to the results that are to be presented. The created system is divided into two interconnected parts aimed to perform segmentation in a slightly different manner. The components of both parts of the system are exactly the same and the only difference, and yet a substantial one, lies in a way a feature function needed for energy calculation is defined. The first part of the system, which is described in details in \textit{chapter \ref{chapter:linear}} is based on the simplest way of expressing features, which is by using a feature vector that represent low level features of separate pixels in an image. On the other hand, the second part of the system involves a higher level of abstraction as instead of directly operating on features, a probability estimator is introduced as a feature function. The details of this part of the system are provided in \textit{chapter \ref{chapter:nonlinear}: \nameref{chapter:nonlinear}}.