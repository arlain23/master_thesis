For semantic image segmentation, far more complex methods are required than for ordinary segmentation. As in case of any segmentation, there is a need of feature extraction and selection that will allow the distinction between objects. However, apart from specifying object boundaries, those features have to carry enough information to differentiate between objects on a higher level, in order to assign the semantic meaning to them. A conventional approach to this task would be based on manual selection of features, which may include previously mentioned colour intensities or detected edges. However, in practice more sophisticated algorithms for feature extraction are preferred. Instead of describing an image as a whole, some areas that are significantly different from others are detected. Those areas are named regions of interest, as they are the ones on which the segmentation process is focused. Then, for every such region a vector of important features is constructed with the use of feature extraction algorithms. The aim of those algorithms is to take an image as an input and encode information from regions of interest into a series of numbers describing a specific region. They take into consideration features like local spatial information, gradients, histograms, object orientation or texture. Only with a proper selection of features, it will be possible for the segmentation algorithm to not only differentiate one region from the other but also to assign a proper class to each region. For natural images, it is an extremely difficult task as the same object may come in various forms, sizes, shapes, colours etc \cite{segmentation_methods_descriptors_2}. 

Basing on the extracted feature vectors classification algorithms conduct semantic image segmentation. Traditionally, algorithms like Support Vector Machines or Random Decision Forests were involved in this task. Though there are relatively old algorithms, as both were proposed around the 1990s \cite{decision_forests,Cortes1995}, they are still applicable and used for both, classification and regression tasks. Support Vector Machine is a supervised learning algorithm that performs classification based on defining decision boundaries between classes. When it comes to Random Decision Forests, they as well involve supervised machine learning to solve classification and regression tasks. This algorithm is based on composing a classifier ensemble from a set of decision trees which are independent of each other in a decision making process. It is done in order to obtain a more accurate and robust prediction comparing to the case of a single and complex classifier. Even though those traditional methods are still applicable for classification tasks, lately more advanced, deep learning methods have become dominant in this field.

Though, initial approaches to deep learning for the described task date back to 2000s \cite{deep_learning_Ciresan}, a breakthrough of using Deep Neural Networks for image segmentation happened in 2014 \cite{cnn_shelhamer} when such a method of using Fully Convolutional Networks was introduced that allowed image inputs of arbitrary size as opposed to first implementations that required the size to be fixed. Since then, most of the algorithms used for semantic image segmentation have been based on this method. Later in the same year \cite{cnn_liang}, a method combining Fully Convolutional Networks with Conditional Random Fields was proposed, which improved the results by better boundary detection and capturing of details. However, Conditional Random Fields, which will be further explained in this dissertation, can themselves act as an independent classifier that can be used to perform semantic image segmentation. 
