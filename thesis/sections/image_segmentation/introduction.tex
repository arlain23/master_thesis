As the main goal of the thesis is to perform semantic segmentation of images, this chapter will be focused on providing an explanation of this task and description of methods that can be used to achieve it. 

In order to present the concept of semantic image segmentation, there is a need to define a more general term, which is image segmentation \cite{digital_image_processing}. It is a process aimed to divide an image into regions that share similar properties, which usually means finding distinct objects and boundaries between them. Depending on an application, those partitions will differ, as the goal is to identify regions that provide relevant data for a given problem and this relevance is specific to the given applications. Segmentation is often conducted at the preprocessing stage as by grouping pixels into larger regions, further analysis is much simpler and less costly in terms of computations. 
There are many techniques to tackle the problem of segmentation, however, most of them can be described by one of three categories: threshold, edge-based and region-based methods \cite{Glasbey_segmentation}. Thresholding is the simplest approach to image segmentation in which individual pixels are assigned to a certain region depending on whether their intensity is in a range of intensities of this region. Edge-based methods are aimed to detect pixels that can be identified as edges between objects, which are characterised by a noticeable change in intensity between neighbouring regions. With all edges detected, it is possible to specify boundaries between objects, giving a full segmentation of an image. The last set of techniques, which are region-based methods, are aimed to group pixels that have similar properties into sets called regions, which correspond to objects or parts of objects present on an image. This may be obtained by region splitting, which means dividing regions into subregions until all the pixels in a region are homogeneous in terms of their features. An alternative process is named region growing, which works by joining neighbouring subregions if they share similar properties in order to obtain larger regions. Very often also a hybrid of both algorithms is used. 

The presented three methods are the most basic ones when it comes to simple segmentation tasks. Even though, for more complicated tasks they are usually not applicable, they can be partially  incorporated into more sophisticated algorithms. 
