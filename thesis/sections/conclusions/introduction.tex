The purpose of this thesis was to develop a system with which it will be possible to perform semantic image segmentation with use of Conditional Random Fields. Before this system was created there was an excessive theoretical research conducted on what are the necessary steps and algorithms needed to implement such a system from scratch. The dissertation was mostly based on a book \textit{"Structured Learning and Prediction in Computer Vision"} by Sebastian Nowozin and Christoph H. Lampert and an article \textit{"TextonBoost for Image Understanding: Multi-Class Object Recognition and Segmentation by Jointly Modeling Texture, Layout, and Context"} written by Jamie Shotton et al. 

In this dissertation two sets of experiments were described. The first one was aimed to perform semantic segmentation on images into three classes, one for objects in the shades of red, second one for bluish regions and the third one for greenish. In the second sets of experiments, objects were to be detected also by shape. There was one extra class introduced representing reddish objects with a shape of letter H, which should be distinguished from other red regions. 

There were three main procedures required to achieve the task described in this dissertation. First of all, images had to be modelled in a way that makes further processing possible. A chosen method for this step was a factorisation process, which represented an image in terms of an undirected graph composed of two types of nodes, input nodes which decode image features, and output nodes, which are used to model the labelling of the system. Features chosen to perform the experiments described in this dissertation were solely based on colour, though they included also contextual information about colours in the neighbourhood of the processed region. For the second type of experiments an automatic feature selection based on stepwise regression was implemented that allowed to choose an optimal set of features with which data from images was modelled. 

The next core part of the developed system was aimed to train parameters of the established model. For this part Stochastic Gradient Descent method was chosen. 